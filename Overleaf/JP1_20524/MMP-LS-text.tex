\cite{ahmad2016optimization}Mustafaiz et al. investigate L18 Orthogonal Array (OA) tests were conducted using input factors such as Peak Current, Duty Cycle, and Voltage Gap. Studies were conducted to determine how machining parameters impacted responses like MRR . \\

\cite{aich2014modeling}Ushasta Aich et al In this study, support vector machines, and PSO are employed to create EDM modeling frameworks. Models for MRR and Ra are created using SVM. To confirm the models' correctness and applicability, testing data sets are used to evaluate them .\\

\cite{balasubramaniyan2021wire}Balsubramaniam et al.focus the major focus of this study is wire electric discharge machining (W-EDM), which is used in SMA. When analyzing machining parameters, such as surface and material removal, current, servo voltage, and pulse on time were taken into account. pulse-free time. Parametric analysis was completed after the response surface approach-based central composite design .\\
\cite{das2014prediction}Milan Kumar Das et al. \cite{} investigate an ANN model has been created by the authors to forecast the surface roughness of EN 31 steel. Average roughness (Ra) is the output neuron, whereas machining parameters are the input neurons. A CCD serves to conduct experiments. Based on performance indicators, the best network is chosen after comparing several training techniques. Results from the L-M algorithm are good. When predicting surface roughness, the ANN model outperforms the RSM model. To examine how process variables affect results, 3D surface plots are employed\\ .
\cite{giusti2020image}Allesandro Giusti et al. investigate a Convolutional Neural Network is employed in a cheap optical measurement system that is combined with an EDM machine to forecast Ra values. Experimental results show that predictions made at different roughness levels are accurate. This is an effective method for characterizing and controlling the roughness of surfaces in machining operations .\\
\cite{goyal2021optimization}Ashish Goyal et al. use surface roughness optimization on an EDM machine. The Taguchi approach has optimized the results that were achieved. ANOVA analysis reveals important criteria for enhancing surface roughness \\.
\cite{joshi2020edm}Anurag et al. focus on the EN8 material utilized on the EDM machine in this study performs better than typical mild steel, which has carbon levels between 0.3 and 0.6\%. For higher values of the process parameters, the MRR is proportional to the EWR Electrode wear rate .\\
\cite{naresh2020artificial}Naresh et al focus on Levenberg-Marquardt (LM) algorithms. It was discovered that LM with 10 neurons was the ideal algorithm and number of neurons in the hidden layer for ANN models .\\
\cite{park2016machine}Je-Kang Park et al.focus on artificial intelligence (AI) and smart sensors to automatically detect surface flaws in components. The suggested technique uses CNN-based image processing to find wear, scratches, and burrs. The efficiency of a single CNN network for assessing various sorts of flaws on textured and non-textured surfaces was demonstrated through the construction and testing of several neural networks.\\
\cite{patel2009determination}R.K.Patel et al;,focus on  Sorface roughness and EDM machining parameters Ton, Vgap, Duty cycle. Using Response surface Method  Ton is found o be dominant parameters.\\
\cite{patel2019texture}Dhiren Patel et al. focus on  machine learning and image processing techniques are used to identify surface quality from machining surfaces. It offers a technique that uses machine learning algorithms to characterize collected photos by extracting statistical information from them. Results with the ANN and RF algorithms demonstrate remarkable accuracy. This study provides a practical method for evaluating machined surfaces' quality, which is advantageous to sectors including manufacturing and quality control .\\
\cite{paturi2021machine}Uma Maheswari,Reddy Patauri et al. Surface roughness significantly improves by 61.31 when the GA technique is used. These outcomes represent quick and precise WEDM of Inconel 718 prediction and optimization approaches .\\
\cite{paul2019multi}T. R. Paul et al. investigate the optimization process parameters in Inconel 800 EDM using a hybrid approach. It is very simple to use MOORA, or multi-objective optimization on the basis of ratio analysis, and it is also very simple mathematically.\\
\cite{routara2020investigation}outara et al. focus on the EDM machining properties of T6-Al7075. For both rotary and steady tool modes, the parameters Tonne, Toff, Ip, and voltage are used. MRR TWR, Ra, and Rq optimization for responses has been examined .\\
\cite{saeedi2021measurement}Jamal Seedi et al. investigate the research describes the use of deep neural networks in an industrial measurement system for determining surface roughness and fault detection on degraded steelwork parts. The suggested techniques are superior to others.  By combining CNN-based regression with CNN-based classification, we achieve accurate roughness estimation (7.32\% error), high defect detection accuracy (97.26\%), and precise localization (99.09\% area under the ROC curve) \\
\cite{shivanna2014evaluation}Shivanna et al use this study compares 3D surface roughness metrics using a confocal and CCD camera using aluminum as a specimen metal. The vision approach is a revolutionary technique.\\
\cite{srikanth2021optimization}R. Srikant Optimization of process param-eters in electric discharge machining process of ti-6al-4v alloy using hybrid taguchi based moora method.\\
\cite{singh2022machine}Ranjit Singh et al In this study, EDM settings for a Cu-based shape memory alloy are optimized using machine learning techniques. The study examines how dimensional deviation and tool wear rate are affected by process parameters. The study makes use of a central composite design matrix and employs 2-D and 3-D graphs to illustrate response parameter behavior. Cu-based SMA machining in EDM processes and optimization by Genetic Algorithm, and Teacher Learning-based optimization approaches both involve machine learning .\\
\cite{ulas2020surface} Mustafa Ulas et al.focus on Aluminum alloys are precisely machined using WEDM and can estimate surface roughness, saving time and money compared to experimentation. Al7075 aluminum alloy experiments were conducted with various WEDM parameters, and machine-learning models for surface roughness prediction were built. The model has the accuracy of 0.9720 and has the most potential for use in manufacturing WEDM-produced parts .\\
\cite{vakharia2022experimental}Vinay Vakharia et al. focus on Nitinol in manufacturing for biomedical and aeronautical applications is examined in this research. The difficulty of producing desired surface characteristics in machined components is discussed in the paper. FESEM (Electro Microscope) is utilized to analyze the surfaces of Nitinol specimens after wiring electrical discharge machining. Surface morphology and its link to process parameters are predicted using deep learning models, such as SinGAN and DenseNet. A useful tool for surface image prediction in manufacturing, the DenseNet model has great accuracy .\\
\cite{varol2021estimating}Hatice Varol et al use the metal alloy Inconel 718 have excellent mechanical, corrosion-resistance, and high-temperature performance characteristics. However, problems arise from their low machinability and the necessity for expensive tools in conventional machining techniques. For cutting hard materials, non-traditional production techniques like electrical discharge machining offer an affordable solution. An artificial intelligence model for evaluating surface roughness based on process parameters was developed using ANN and GEP .\\
\cite{yogesh2021predicton}Yogesh et al. draw attention to the difficulties in determining the ideal parameters for the highest MRR and the lowest Ra as well as the limitations of conventional testing techniques. The suggested approach uses Decision Tree and Naive Bayes algorithms to forecast SR and MRR, saving time and important resources. The emphasis is especially on the EDM machining of aluminum composites, demonstrating how the algorithms can be used in this situation. \\
